%% Generated by Sphinx.
\def\sphinxdocclass{report}
\documentclass[a4paper,11pt,spanish]{sphinxmanual}
\ifdefined\pdfpxdimen
   \let\sphinxpxdimen\pdfpxdimen\else\newdimen\sphinxpxdimen
\fi \sphinxpxdimen=49336sp\relax

\usepackage[margin=1in,marginparwidth=0.5in]{geometry}

\catcode`^^^^00a0\active\protected\def^^^^00a0{\leavevmode\nobreak\ }
\usepackage{cmap}
\usepackage{fontspec}
\usepackage{amsmath,amssymb,amstext}
\usepackage{polyglossia}
\setmainlanguage{spanish}
\usepackage{libertine}
                  \setmonofont[
                     Path=/home/ricardo/fonts/,
                     BoldFont=Inconsolata-Bold.ttf,
                     AutoFakeSlant,
                     BoldItalicFeatures={FakeSlant},
                     Scale=MatchLowercase
                  ]{Inconsolata-Regular.ttf}
\usepackage[Sonny]{fncychap}
\usepackage{longtable}
\usepackage{sphinx}
\sphinxsetup{verbatimwithframe=false}
\usepackage{multirow}
\usepackage{eqparbox}

% Include hyperref last.
\usepackage{hyperref}
% Fix anchor placement for figures with captions.
\usepackage{hypcap}% it must be loaded after hyperref.
% Set up styles of URL: it should be placed after hyperref.
\urlstyle{same}

\addto\captionsspanish{\renewcommand{\figurename}{Figura}}
\addto\captionsspanish{\renewcommand{\tablename}{Tabla}}
\addto\captionsspanish{\renewcommand{\literalblockname}{Lista}}

\def\pageautorefname{página}

\setcounter{tocdepth}{0}

\usepackage[dotinlabels]{titletoc}
                    \usepackage{titlesec}
                    \titlelabel{\thetitle.\quad}

\title{Desarrollo web en entorno servidor}
\date{febrero de 2017}
\release{1.0}
\author{Ricardo Pérez López}
\newcommand{\sphinxlogo}{}
\renewcommand{\releasename}{Versión}
\makeindex

\begin{document}

\maketitle
\sphinxtableofcontents
\phantomsection\label{\detokenize{index::doc}}


Contents:


\part{Overview}
\label{\detokenize{overview:overview}}\label{\detokenize{overview::doc}}\label{\detokenize{overview:welcome-to-desarrollo-web-en-entorno-servidor-s-documentation}}

\chapter{Requirements}
\label{\detokenize{overview:requirements}}\begin{enumerate}
\item {} 
PHP 5.5.0

\item {} 
To use the PHP stream handler, \sphinxcode{allow\_url\_fopen} must be enabled in your
system's php.ini.

\item {} 
To use the cURL handler, you must have a recent version of cURL \textgreater{}= 7.19.4
compiled with OpenSSL and zlib.

\end{enumerate}

\begin{sphinxadmonition}{note}{Nota:}
Guzzle no longer requires cURL in order to send HTTP requests. Guzzle will
use the PHP stream wrapper to send HTTP requests if cURL is not installed.
Alternatively, you can provide your own HTTP handler used to send requests.
\end{sphinxadmonition}


\chapter{Installation}
\label{\detokenize{overview:installation}}\label{\detokenize{overview:id1}}
The recommended way to install Guzzle is with
\href{http://getcomposer.org}{Composer}. Composer is a dependency management tool
for PHP that allows you to declare the dependencies your project needs and
installs them into your project.

\begin{sphinxVerbatim}[commandchars=\\\{\}]
\PYG{c+c1}{\PYGZsh{} Install Composer}
curl \PYGZhy{}sS https://getcomposer.org/installer \PYG{p}{\textbar{}} php
\end{sphinxVerbatim}

You can add Guzzle as a dependency using the composer.phar CLI:

\begin{sphinxVerbatim}[commandchars=\\\{\}]
php composer.phar require guzzlehttp/guzzle:\PYGZti{}6.0
\end{sphinxVerbatim}

Alternatively, you can specify Guzzle as a dependency in your project's
existing composer.json file:

\begin{sphinxVerbatim}[commandchars=\\\{\}]
 \PYG{p}{\PYGZob{}}
   \PYG{l+s+s2}{\PYGZdq{}require\PYGZdq{}}\PYG{o}{:} \PYG{p}{\PYGZob{}}
      \PYG{l+s+s2}{\PYGZdq{}guzzlehttp/guzzle\PYGZdq{}}\PYG{o}{:} \PYG{l+s+s2}{\PYGZdq{}\PYGZti{}6.0\PYGZdq{}}
   \PYG{p}{\PYGZcb{}}
\PYG{p}{\PYGZcb{}}
\end{sphinxVerbatim}

After installing, you need to require Composer's autoloader:

\begin{sphinxVerbatim}[commandchars=\\\{\}]
\PYG{k}{require} \PYG{l+s+s1}{\PYGZsq{}vendor/autoload.php\PYGZsq{}}\PYG{p}{;}
\end{sphinxVerbatim}

You can find out more on how to install Composer, configure autoloading, and
other best-practices for defining dependencies at \href{http://getcomposer.org}{getcomposer.org}.


\section{Bleeding edge}
\label{\detokenize{overview:bleeding-edge}}
During your development, you can keep up with the latest changes on the master
branch by setting the version requirement for Guzzle to \sphinxcode{\textasciitilde{}6.0@dev}.

\begin{sphinxVerbatim}[commandchars=\\\{\}]
\PYG{p}{\PYGZob{}}
   \PYG{l+s+s2}{\PYGZdq{}require\PYGZdq{}}\PYG{o}{:} \PYG{p}{\PYGZob{}}
      \PYG{l+s+s2}{\PYGZdq{}guzzlehttp/guzzle\PYGZdq{}}\PYG{o}{:} \PYG{l+s+s2}{\PYGZdq{}\PYGZti{}6.0@dev\PYGZdq{}}
   \PYG{p}{\PYGZcb{}}
\PYG{p}{\PYGZcb{}}
\end{sphinxVerbatim}


\chapter{License}
\label{\detokenize{overview:license}}
Licensed using the \href{http://opensource.org/licenses/MIT}{MIT license}.
\begin{quote}

Copyright (c) 2015 Michael Dowling \textless{}\url{https://github.com/mtdowling}\textgreater{}

Permission is hereby granted, free of charge, to any person obtaining a copy
of this software and associated documentation files (the ``Software''), to deal
in the Software without restriction, including without limitation the rights
to use, copy, modify, merge, publish, distribute, sublicense, and/or sell
copies of the Software, and to permit persons to whom the Software is
furnished to do so, subject to the following conditions:

The above copyright notice and this permission notice shall be included in
all copies or substantial portions of the Software.

THE SOFTWARE IS PROVIDED ``AS IS'', WITHOUT WARRANTY OF ANY KIND, EXPRESS OR
IMPLIED, INCLUDING BUT NOT LIMITED TO THE WARRANTIES OF MERCHANTABILITY,
FITNESS FOR A PARTICULAR PURPOSE AND NONINFRINGEMENT. IN NO EVENT SHALL THE
AUTHORS OR COPYRIGHT HOLDERS BE LIABLE FOR ANY CLAIM, DAMAGES OR OTHER
LIABILITY, WHETHER IN AN ACTION OF CONTRACT, TORT OR OTHERWISE, ARISING FROM,
OUT OF OR IN CONNECTION WITH THE SOFTWARE OR THE USE OR OTHER DEALINGS IN
THE SOFTWARE.
\end{quote}


\chapter{Contributing}
\label{\detokenize{overview:contributing}}

\section{Guidelines}
\label{\detokenize{overview:guidelines}}\begin{enumerate}
\item {} 
Guzzle utilizes PSR-1, PSR-2, PSR-4, and PSR-7.

\item {} 
Guzzle is meant to be lean and fast with very few dependencies. This means
that not every feature request will be accepted.

\item {} 
Guzzle has a minimum PHP version requirement of PHP 5.5. Pull requests must
not require a PHP version greater than PHP 5.5 unless the feature is only
utilized conditionally.

\item {} 
All pull requests must include unit tests to ensure the change works as
expected and to prevent regressions.

\end{enumerate}


\section{Running the tests}
\label{\detokenize{overview:running-the-tests}}
In order to contribute, you'll need to checkout the source from GitHub and
install Guzzle's dependencies using Composer:

\begin{sphinxVerbatim}[commandchars=\\\{\}]
git clone https://github.com/guzzle/guzzle.git
\PYG{n+nb}{cd} guzzle \PYG{o}{\PYGZam{}\PYGZam{}} curl \PYGZhy{}s http://getcomposer.org/installer \PYG{p}{\textbar{}} php \PYG{o}{\PYGZam{}\PYGZam{}} ./composer.phar install \PYGZhy{}\PYGZhy{}dev
\end{sphinxVerbatim}

Guzzle is unit tested with PHPUnit. Run the tests using the Makefile:

\begin{sphinxVerbatim}[commandchars=\\\{\}]
make \PYG{n+nb}{test}
\end{sphinxVerbatim}

\begin{sphinxadmonition}{note}{Nota:}
You'll need to install node.js v0.5.0 or newer in order to perform
integration tests on Guzzle's HTTP handlers.
\end{sphinxadmonition}


\chapter{Reporting a security vulnerability}
\label{\detokenize{overview:reporting-a-security-vulnerability}}
We want to ensure that Guzzle is a secure HTTP client library for everyone. If
you've discovered a security vulnerability in Guzzle, we appreciate your help
in disclosing it to us in a \href{http://en.wikipedia.org/wiki/Responsible\_disclosure}{responsible manner}.

Publicly disclosing a vulnerability can put the entire community at risk. If
you've discovered a security concern, please email us at
\href{mailto:security@guzzlephp.org}{security@guzzlephp.org}. We'll work with you to make sure that we understand the
scope of the issue, and that we fully address your concern. We consider
correspondence sent to \href{mailto:security@guzzlephp.org}{security@guzzlephp.org} our highest priority, and work to
address any issues that arise as quickly as possible.

After a security vulnerability has been corrected, a security hotfix release will
be deployed as soon as possible.


\part{Quickstart}
\label{\detokenize{quickstart::doc}}\label{\detokenize{quickstart:quickstart}}
This page provides a quick introduction to Guzzle and introductory examples.
If you have not already installed, Guzzle, head over to the {\hyperref[\detokenize{overview:installation}]{\sphinxcrossref{\DUrole{std,std-ref}{Installation}}}}
page.


\chapter{Making a Request}
\label{\detokenize{quickstart:making-a-request}}
You can send requests with Guzzle using a \sphinxcode{GuzzleHttp\textbackslash{}ClientInterface}
object.


\section{Creating a Client}
\label{\detokenize{quickstart:creating-a-client}}
\begin{sphinxVerbatim}[commandchars=\\\{\}]
\PYG{k}{use} \PYG{n+nx}{GuzzleHttp\PYGZbs{}Client}\PYG{p}{;}

\PYG{n+nv}{\PYGZdl{}client} \PYG{o}{=} \PYG{k}{new} \PYG{n+nx}{Client}\PYG{p}{([}
    \PYG{c+c1}{// Base URI is used with relative requests}
    \PYG{l+s+s1}{\PYGZsq{}base\PYGZus{}uri\PYGZsq{}} \PYG{o}{=\PYGZgt{}} \PYG{l+s+s1}{\PYGZsq{}http://httpbin.org\PYGZsq{}}\PYG{p}{,}
    \PYG{c+c1}{// You can set any number of default request options.}
    \PYG{l+s+s1}{\PYGZsq{}timeout\PYGZsq{}}  \PYG{o}{=\PYGZgt{}} \PYG{l+m+mf}{2.0}\PYG{p}{,}
\PYG{p}{]);}
\end{sphinxVerbatim}

Clients are immutable in Guzzle 6, which means that you cannot change the defaults used by a client after it's created.

The client constructor accepts an associative array of options:
\begin{description}
\item[{\sphinxcode{base\_uri}}] \leavevmode
(string\textbar{}UriInterface) Base URI of the client that is merged into relative
URIs. Can be a string or instance of UriInterface. When a relative URI
is provided to a client, the client will combine the base URI with the
relative URI using the rules described in
\href{http://tools.ietf.org/html/rfc3986\#section-5.2}{RFC 3986, section 2}.

\begin{sphinxVerbatim}[commandchars=\\\{\}]
\PYG{c+c1}{// Create a client with a base URI}
\PYG{n+nv}{\PYGZdl{}client} \PYG{o}{=} \PYG{k}{new} \PYG{n+nx}{GuzzleHttp\PYGZbs{}Client}\PYG{p}{([}\PYG{l+s+s1}{\PYGZsq{}base\PYGZus{}uri\PYGZsq{}} \PYG{o}{=\PYGZgt{}} \PYG{l+s+s1}{\PYGZsq{}https://foo.com/api/\PYGZsq{}}\PYG{p}{]);}
\PYG{c+c1}{// Send a request to https://foo.com/api/test}
\PYG{n+nv}{\PYGZdl{}response} \PYG{o}{=} \PYG{n+nv}{\PYGZdl{}client}\PYG{o}{\PYGZhy{}\PYGZgt{}}\PYG{n+na}{request}\PYG{p}{(}\PYG{l+s+s1}{\PYGZsq{}GET\PYGZsq{}}\PYG{p}{,} \PYG{l+s+s1}{\PYGZsq{}test\PYGZsq{}}\PYG{p}{);}
\PYG{c+c1}{// Send a request to https://foo.com/root}
\PYG{n+nv}{\PYGZdl{}response} \PYG{o}{=} \PYG{n+nv}{\PYGZdl{}client}\PYG{o}{\PYGZhy{}\PYGZgt{}}\PYG{n+na}{request}\PYG{p}{(}\PYG{l+s+s1}{\PYGZsq{}GET\PYGZsq{}}\PYG{p}{,} \PYG{l+s+s1}{\PYGZsq{}/root\PYGZsq{}}\PYG{p}{);}
\end{sphinxVerbatim}

Don't feel like reading RFC 3986? Here are some quick examples on how a
\sphinxcode{base\_uri} is resolved with another URI.

\noindent\begin{tabulary}{\linewidth}{|L|L|L|}
\hline
\sphinxstylethead{\relax 
base\_uri
\unskip}\relax &\sphinxstylethead{\relax 
URI
\unskip}\relax &\sphinxstylethead{\relax 
Result
\unskip}\relax \\
\hline
\sphinxcode{http://foo.com}
&
\sphinxcode{/bar}
&
\sphinxcode{http://foo.com/bar}
\\
\hline
\sphinxcode{http://foo.com/foo}
&
\sphinxcode{/bar}
&
\sphinxcode{http://foo.com/bar}
\\
\hline
\sphinxcode{http://foo.com/foo}
&
\sphinxcode{bar}
&
\sphinxcode{http://foo.com/bar}
\\
\hline
\sphinxcode{http://foo.com/foo/}
&
\sphinxcode{bar}
&
\sphinxcode{http://foo.com/foo/bar}
\\
\hline
\sphinxcode{http://foo.com}
&
\sphinxcode{http://baz.com}
&
\sphinxcode{http://baz.com}
\\
\hline
\sphinxcode{http://foo.com/?bar}
&
\sphinxcode{bar}
&
\sphinxcode{http://foo.com/bar}
\\
\hline\end{tabulary}


\item[{\sphinxcode{handler}}] \leavevmode
(callable) Function that transfers HTTP requests over the wire. The
function is called with a \sphinxcode{Psr7\textbackslash{}Http\textbackslash{}Message\textbackslash{}RequestInterface} and array
of transfer options, and must return a
\sphinxcode{GuzzleHttp\textbackslash{}Promise\textbackslash{}PromiseInterface} that is fulfilled with a
\sphinxcode{Psr7\textbackslash{}Http\textbackslash{}Message\textbackslash{}ResponseInterface} on success. \sphinxcode{handler} is a
constructor only option that cannot be overridden in per/request options.

\item[{\sphinxcode{...}}] \leavevmode
(mixed) All other options passed to the constructor are used as default
request options with every request created by the client.

\end{description}


\section{Sending Requests}
\label{\detokenize{quickstart:sending-requests}}
Magic methods on the client make it easy to send synchronous requests:

\begin{sphinxVerbatim}[commandchars=\\\{\}]
\PYG{n+nv}{\PYGZdl{}response} \PYG{o}{=} \PYG{n+nv}{\PYGZdl{}client}\PYG{o}{\PYGZhy{}\PYGZgt{}}\PYG{n+na}{get}\PYG{p}{(}\PYG{l+s+s1}{\PYGZsq{}http://httpbin.org/get\PYGZsq{}}\PYG{p}{);}
\PYG{n+nv}{\PYGZdl{}response} \PYG{o}{=} \PYG{n+nv}{\PYGZdl{}client}\PYG{o}{\PYGZhy{}\PYGZgt{}}\PYG{n+na}{delete}\PYG{p}{(}\PYG{l+s+s1}{\PYGZsq{}http://httpbin.org/delete\PYGZsq{}}\PYG{p}{);}
\PYG{n+nv}{\PYGZdl{}response} \PYG{o}{=} \PYG{n+nv}{\PYGZdl{}client}\PYG{o}{\PYGZhy{}\PYGZgt{}}\PYG{n+na}{head}\PYG{p}{(}\PYG{l+s+s1}{\PYGZsq{}http://httpbin.org/get\PYGZsq{}}\PYG{p}{);}
\PYG{n+nv}{\PYGZdl{}response} \PYG{o}{=} \PYG{n+nv}{\PYGZdl{}client}\PYG{o}{\PYGZhy{}\PYGZgt{}}\PYG{n+na}{options}\PYG{p}{(}\PYG{l+s+s1}{\PYGZsq{}http://httpbin.org/get\PYGZsq{}}\PYG{p}{);}
\PYG{n+nv}{\PYGZdl{}response} \PYG{o}{=} \PYG{n+nv}{\PYGZdl{}client}\PYG{o}{\PYGZhy{}\PYGZgt{}}\PYG{n+na}{patch}\PYG{p}{(}\PYG{l+s+s1}{\PYGZsq{}http://httpbin.org/patch\PYGZsq{}}\PYG{p}{);}
\PYG{n+nv}{\PYGZdl{}response} \PYG{o}{=} \PYG{n+nv}{\PYGZdl{}client}\PYG{o}{\PYGZhy{}\PYGZgt{}}\PYG{n+na}{post}\PYG{p}{(}\PYG{l+s+s1}{\PYGZsq{}http://httpbin.org/post\PYGZsq{}}\PYG{p}{);}
\PYG{n+nv}{\PYGZdl{}response} \PYG{o}{=} \PYG{n+nv}{\PYGZdl{}client}\PYG{o}{\PYGZhy{}\PYGZgt{}}\PYG{n+na}{put}\PYG{p}{(}\PYG{l+s+s1}{\PYGZsq{}http://httpbin.org/put\PYGZsq{}}\PYG{p}{);}
\end{sphinxVerbatim}

You can create a request and then send the request with the client when you're
ready:

\begin{sphinxVerbatim}[commandchars=\\\{\}]
\PYG{k}{use} \PYG{n+nx}{GuzzleHttp\PYGZbs{}Psr7\PYGZbs{}Request}\PYG{p}{;}

\PYG{n+nv}{\PYGZdl{}request} \PYG{o}{=} \PYG{k}{new} \PYG{n+nx}{Request}\PYG{p}{(}\PYG{l+s+s1}{\PYGZsq{}PUT\PYGZsq{}}\PYG{p}{,} \PYG{l+s+s1}{\PYGZsq{}http://httpbin.org/put\PYGZsq{}}\PYG{p}{);}
\PYG{n+nv}{\PYGZdl{}response} \PYG{o}{=} \PYG{n+nv}{\PYGZdl{}client}\PYG{o}{\PYGZhy{}\PYGZgt{}}\PYG{n+na}{send}\PYG{p}{(}\PYG{n+nv}{\PYGZdl{}request}\PYG{p}{,} \PYG{p}{[}\PYG{l+s+s1}{\PYGZsq{}timeout\PYGZsq{}} \PYG{o}{=\PYGZgt{}} \PYG{l+m+mi}{2}\PYG{p}{]);}
\end{sphinxVerbatim}

Client objects provide a great deal of flexibility in how request are
transferred including default request options, default handler stack middleware
that are used by each request, and a base URI that allows you to send requests
with relative URIs.

You can find out more about client middleware in the
\DUrole{xref,doc}{handlers-and-middleware} page of the documentation.


\section{Async Requests}
\label{\detokenize{quickstart:async-requests}}
You can send asynchronous requests using the magic methods provided by a client:

\begin{sphinxVerbatim}[commandchars=\\\{\}]
\PYG{n+nv}{\PYGZdl{}promise} \PYG{o}{=} \PYG{n+nv}{\PYGZdl{}client}\PYG{o}{\PYGZhy{}\PYGZgt{}}\PYG{n+na}{getAsync}\PYG{p}{(}\PYG{l+s+s1}{\PYGZsq{}http://httpbin.org/get\PYGZsq{}}\PYG{p}{);}
\PYG{n+nv}{\PYGZdl{}promise} \PYG{o}{=} \PYG{n+nv}{\PYGZdl{}client}\PYG{o}{\PYGZhy{}\PYGZgt{}}\PYG{n+na}{deleteAsync}\PYG{p}{(}\PYG{l+s+s1}{\PYGZsq{}http://httpbin.org/delete\PYGZsq{}}\PYG{p}{);}
\PYG{n+nv}{\PYGZdl{}promise} \PYG{o}{=} \PYG{n+nv}{\PYGZdl{}client}\PYG{o}{\PYGZhy{}\PYGZgt{}}\PYG{n+na}{headAsync}\PYG{p}{(}\PYG{l+s+s1}{\PYGZsq{}http://httpbin.org/get\PYGZsq{}}\PYG{p}{);}
\PYG{n+nv}{\PYGZdl{}promise} \PYG{o}{=} \PYG{n+nv}{\PYGZdl{}client}\PYG{o}{\PYGZhy{}\PYGZgt{}}\PYG{n+na}{optionsAsync}\PYG{p}{(}\PYG{l+s+s1}{\PYGZsq{}http://httpbin.org/get\PYGZsq{}}\PYG{p}{);}
\PYG{n+nv}{\PYGZdl{}promise} \PYG{o}{=} \PYG{n+nv}{\PYGZdl{}client}\PYG{o}{\PYGZhy{}\PYGZgt{}}\PYG{n+na}{patchAsync}\PYG{p}{(}\PYG{l+s+s1}{\PYGZsq{}http://httpbin.org/patch\PYGZsq{}}\PYG{p}{);}
\PYG{n+nv}{\PYGZdl{}promise} \PYG{o}{=} \PYG{n+nv}{\PYGZdl{}client}\PYG{o}{\PYGZhy{}\PYGZgt{}}\PYG{n+na}{postAsync}\PYG{p}{(}\PYG{l+s+s1}{\PYGZsq{}http://httpbin.org/post\PYGZsq{}}\PYG{p}{);}
\PYG{n+nv}{\PYGZdl{}promise} \PYG{o}{=} \PYG{n+nv}{\PYGZdl{}client}\PYG{o}{\PYGZhy{}\PYGZgt{}}\PYG{n+na}{putAsync}\PYG{p}{(}\PYG{l+s+s1}{\PYGZsq{}http://httpbin.org/put\PYGZsq{}}\PYG{p}{);}
\end{sphinxVerbatim}

You can also use the \sphinxtitleref{sendAsync()} and \sphinxtitleref{requestAsync()} methods of a client:

\begin{sphinxVerbatim}[commandchars=\\\{\}]
\PYG{k}{use} \PYG{n+nx}{GuzzleHttp\PYGZbs{}Psr7\PYGZbs{}Request}\PYG{p}{;}

\PYG{c+c1}{// Create a PSR\PYGZhy{}7 request object to send}
\PYG{n+nv}{\PYGZdl{}headers} \PYG{o}{=} \PYG{p}{[}\PYG{l+s+s1}{\PYGZsq{}X\PYGZhy{}Foo\PYGZsq{}} \PYG{o}{=\PYGZgt{}} \PYG{l+s+s1}{\PYGZsq{}Bar\PYGZsq{}}\PYG{p}{];}
\PYG{n+nv}{\PYGZdl{}body} \PYG{o}{=} \PYG{l+s+s1}{\PYGZsq{}Hello!\PYGZsq{}}\PYG{p}{;}
\PYG{n+nv}{\PYGZdl{}request} \PYG{o}{=} \PYG{k}{new} \PYG{n+nx}{Request}\PYG{p}{(}\PYG{l+s+s1}{\PYGZsq{}HEAD\PYGZsq{}}\PYG{p}{,} \PYG{l+s+s1}{\PYGZsq{}http://httpbin.org/head\PYGZsq{}}\PYG{p}{,} \PYG{n+nv}{\PYGZdl{}headers}\PYG{p}{,} \PYG{n+nv}{\PYGZdl{}body}\PYG{p}{);}

\PYG{c+c1}{// Or, if you don\PYGZsq{}t need to pass in a request instance:}
\PYG{n+nv}{\PYGZdl{}promise} \PYG{o}{=} \PYG{n+nv}{\PYGZdl{}client}\PYG{o}{\PYGZhy{}\PYGZgt{}}\PYG{n+na}{requestAsync}\PYG{p}{(}\PYG{l+s+s1}{\PYGZsq{}GET\PYGZsq{}}\PYG{p}{,} \PYG{l+s+s1}{\PYGZsq{}http://httpbin.org/get\PYGZsq{}}\PYG{p}{);}
\end{sphinxVerbatim}

The promise returned by these methods implements the
\href{https://promisesaplus.com/}{Promises/A+ spec}, provided by the
\href{https://github.com/guzzle/promises}{Guzzle promises library}. This means
that you can chain \sphinxcode{then()} calls off of the promise. These then calls are
either fulfilled with a successful \sphinxcode{Psr\textbackslash{}Http\textbackslash{}Message\textbackslash{}ResponseInterface} or
rejected with an exception.

\begin{sphinxVerbatim}[commandchars=\\\{\}]
\PYG{k}{use} \PYG{n+nx}{Psr\PYGZbs{}Http\PYGZbs{}Message\PYGZbs{}ResponseInterface}\PYG{p}{;}
\PYG{k}{use} \PYG{n+nx}{GuzzleHttp\PYGZbs{}Exception\PYGZbs{}RequestException}\PYG{p}{;}

\PYG{n+nv}{\PYGZdl{}promise} \PYG{o}{=} \PYG{n+nv}{\PYGZdl{}client}\PYG{o}{\PYGZhy{}\PYGZgt{}}\PYG{n+na}{requestAsync}\PYG{p}{(}\PYG{l+s+s1}{\PYGZsq{}GET\PYGZsq{}}\PYG{p}{,} \PYG{l+s+s1}{\PYGZsq{}http://httpbin.org/get\PYGZsq{}}\PYG{p}{);}
\PYG{n+nv}{\PYGZdl{}promise}\PYG{o}{\PYGZhy{}\PYGZgt{}}\PYG{n+na}{then}\PYG{p}{(}
    \PYG{k}{function} \PYG{p}{(}\PYG{n+nx}{ResponseInterface} \PYG{n+nv}{\PYGZdl{}res}\PYG{p}{)} \PYG{p}{\PYGZob{}}
        \PYG{k}{echo} \PYG{n+nv}{\PYGZdl{}res}\PYG{o}{\PYGZhy{}\PYGZgt{}}\PYG{n+na}{getStatusCode}\PYG{p}{()} \PYG{o}{.} \PYG{l+s+s2}{\PYGZdq{}}\PYG{l+s+se}{\PYGZbs{}n}\PYG{l+s+s2}{\PYGZdq{}}\PYG{p}{;}
    \PYG{p}{\PYGZcb{},}
    \PYG{k}{function} \PYG{p}{(}\PYG{n+nx}{RequestException} \PYG{n+nv}{\PYGZdl{}e}\PYG{p}{)} \PYG{p}{\PYGZob{}}
        \PYG{k}{echo} \PYG{n+nv}{\PYGZdl{}e}\PYG{o}{\PYGZhy{}\PYGZgt{}}\PYG{n+na}{getMessage}\PYG{p}{()} \PYG{o}{.} \PYG{l+s+s2}{\PYGZdq{}}\PYG{l+s+se}{\PYGZbs{}n}\PYG{l+s+s2}{\PYGZdq{}}\PYG{p}{;}
        \PYG{k}{echo} \PYG{n+nv}{\PYGZdl{}e}\PYG{o}{\PYGZhy{}\PYGZgt{}}\PYG{n+na}{getRequest}\PYG{p}{()}\PYG{o}{\PYGZhy{}\PYGZgt{}}\PYG{n+na}{getMethod}\PYG{p}{();}
    \PYG{p}{\PYGZcb{}}
\PYG{p}{);}
\end{sphinxVerbatim}


\section{Concurrent requests}
\label{\detokenize{quickstart:concurrent-requests}}
You can send multiple requests concurrently using promises and asynchronous
requests.

\begin{sphinxVerbatim}[commandchars=\\\{\}]
\PYG{k}{use} \PYG{n+nx}{GuzzleHttp\PYGZbs{}Client}\PYG{p}{;}
\PYG{k}{use} \PYG{n+nx}{GuzzleHttp\PYGZbs{}Promise}\PYG{p}{;}

\PYG{n+nv}{\PYGZdl{}client} \PYG{o}{=} \PYG{k}{new} \PYG{n+nx}{Client}\PYG{p}{([}\PYG{l+s+s1}{\PYGZsq{}base\PYGZus{}uri\PYGZsq{}} \PYG{o}{=\PYGZgt{}} \PYG{l+s+s1}{\PYGZsq{}http://httpbin.org/\PYGZsq{}}\PYG{p}{]);}

\PYG{c+c1}{// Initiate each request but do not block}
\PYG{n+nv}{\PYGZdl{}promises} \PYG{o}{=} \PYG{p}{[}
    \PYG{l+s+s1}{\PYGZsq{}image\PYGZsq{}} \PYG{o}{=\PYGZgt{}} \PYG{n+nv}{\PYGZdl{}client}\PYG{o}{\PYGZhy{}\PYGZgt{}}\PYG{n+na}{getAsync}\PYG{p}{(}\PYG{l+s+s1}{\PYGZsq{}/image\PYGZsq{}}\PYG{p}{),}
    \PYG{l+s+s1}{\PYGZsq{}png\PYGZsq{}}   \PYG{o}{=\PYGZgt{}} \PYG{n+nv}{\PYGZdl{}client}\PYG{o}{\PYGZhy{}\PYGZgt{}}\PYG{n+na}{getAsync}\PYG{p}{(}\PYG{l+s+s1}{\PYGZsq{}/image/png\PYGZsq{}}\PYG{p}{),}
    \PYG{l+s+s1}{\PYGZsq{}jpeg\PYGZsq{}}  \PYG{o}{=\PYGZgt{}} \PYG{n+nv}{\PYGZdl{}client}\PYG{o}{\PYGZhy{}\PYGZgt{}}\PYG{n+na}{getAsync}\PYG{p}{(}\PYG{l+s+s1}{\PYGZsq{}/image/jpeg\PYGZsq{}}\PYG{p}{),}
    \PYG{l+s+s1}{\PYGZsq{}webp\PYGZsq{}}  \PYG{o}{=\PYGZgt{}} \PYG{n+nv}{\PYGZdl{}client}\PYG{o}{\PYGZhy{}\PYGZgt{}}\PYG{n+na}{getAsync}\PYG{p}{(}\PYG{l+s+s1}{\PYGZsq{}/image/webp\PYGZsq{}}\PYG{p}{)}
\PYG{p}{];}

\PYG{c+c1}{// Wait on all of the requests to complete. Throws a ConnectException}
\PYG{c+c1}{// if any of the requests fail}
\PYG{n+nv}{\PYGZdl{}results} \PYG{o}{=} \PYG{n+nx}{Promise\PYGZbs{}unwrap}\PYG{p}{(}\PYG{n+nv}{\PYGZdl{}promises}\PYG{p}{);}

\PYG{c+c1}{// Wait for the requests to complete, even if some of them fail}
\PYG{n+nv}{\PYGZdl{}results} \PYG{o}{=} \PYG{n+nx}{Promise\PYGZbs{}settle}\PYG{p}{(}\PYG{n+nv}{\PYGZdl{}promises}\PYG{p}{)}\PYG{o}{\PYGZhy{}\PYGZgt{}}\PYG{n+na}{wait}\PYG{p}{();}

\PYG{c+c1}{// You can access each result using the key provided to the unwrap}
\PYG{c+c1}{// function.}
\PYG{k}{echo} \PYG{n+nv}{\PYGZdl{}results}\PYG{p}{[}\PYG{l+s+s1}{\PYGZsq{}image\PYGZsq{}}\PYG{p}{]}\PYG{o}{\PYGZhy{}\PYGZgt{}}\PYG{n+na}{getHeader}\PYG{p}{(}\PYG{l+s+s1}{\PYGZsq{}Content\PYGZhy{}Length\PYGZsq{}}\PYG{p}{);}
\PYG{k}{echo} \PYG{n+nv}{\PYGZdl{}results}\PYG{p}{[}\PYG{l+s+s1}{\PYGZsq{}png\PYGZsq{}}\PYG{p}{]}\PYG{o}{\PYGZhy{}\PYGZgt{}}\PYG{n+na}{getHeader}\PYG{p}{(}\PYG{l+s+s1}{\PYGZsq{}Content\PYGZhy{}Length\PYGZsq{}}\PYG{p}{);}
\end{sphinxVerbatim}

You can use the \sphinxcode{GuzzleHttp\textbackslash{}Pool} object when you have an indeterminate
amount of requests you wish to send.

\begin{sphinxVerbatim}[commandchars=\\\{\}]
\PYG{k}{use} \PYG{n+nx}{GuzzleHttp\PYGZbs{}Pool}\PYG{p}{;}
\PYG{k}{use} \PYG{n+nx}{GuzzleHttp\PYGZbs{}Client}\PYG{p}{;}
\PYG{k}{use} \PYG{n+nx}{GuzzleHttp\PYGZbs{}Psr7\PYGZbs{}Request}\PYG{p}{;}

\PYG{n+nv}{\PYGZdl{}client} \PYG{o}{=} \PYG{k}{new} \PYG{n+nx}{Client}\PYG{p}{();}

\PYG{n+nv}{\PYGZdl{}requests} \PYG{o}{=} \PYG{k}{function} \PYG{p}{(}\PYG{n+nv}{\PYGZdl{}total}\PYG{p}{)} \PYG{p}{\PYGZob{}}
    \PYG{n+nv}{\PYGZdl{}uri} \PYG{o}{=} \PYG{l+s+s1}{\PYGZsq{}http://127.0.0.1:8126/guzzle\PYGZhy{}server/perf\PYGZsq{}}\PYG{p}{;}
    \PYG{k}{for} \PYG{p}{(}\PYG{n+nv}{\PYGZdl{}i} \PYG{o}{=} \PYG{l+m+mi}{0}\PYG{p}{;} \PYG{n+nv}{\PYGZdl{}i} \PYG{o}{\PYGZlt{}} \PYG{n+nv}{\PYGZdl{}total}\PYG{p}{;} \PYG{n+nv}{\PYGZdl{}i}\PYG{o}{++}\PYG{p}{)} \PYG{p}{\PYGZob{}}
        \PYG{k}{yield} \PYG{k}{new} \PYG{n+nx}{Request}\PYG{p}{(}\PYG{l+s+s1}{\PYGZsq{}GET\PYGZsq{}}\PYG{p}{,} \PYG{n+nv}{\PYGZdl{}uri}\PYG{p}{);}
    \PYG{p}{\PYGZcb{}}
\PYG{p}{\PYGZcb{};}

\PYG{n+nv}{\PYGZdl{}pool} \PYG{o}{=} \PYG{k}{new} \PYG{n+nx}{Pool}\PYG{p}{(}\PYG{n+nv}{\PYGZdl{}client}\PYG{p}{,} \PYG{n+nv}{\PYGZdl{}requests}\PYG{p}{(}\PYG{l+m+mi}{100}\PYG{p}{),} \PYG{p}{[}
    \PYG{l+s+s1}{\PYGZsq{}concurrency\PYGZsq{}} \PYG{o}{=\PYGZgt{}} \PYG{l+m+mi}{5}\PYG{p}{,}
    \PYG{l+s+s1}{\PYGZsq{}fulfilled\PYGZsq{}} \PYG{o}{=\PYGZgt{}} \PYG{k}{function} \PYG{p}{(}\PYG{n+nv}{\PYGZdl{}response}\PYG{p}{,} \PYG{n+nv}{\PYGZdl{}index}\PYG{p}{)} \PYG{p}{\PYGZob{}}
        \PYG{c+c1}{// this is delivered each successful response}
    \PYG{p}{\PYGZcb{},}
    \PYG{l+s+s1}{\PYGZsq{}rejected\PYGZsq{}} \PYG{o}{=\PYGZgt{}} \PYG{k}{function} \PYG{p}{(}\PYG{n+nv}{\PYGZdl{}reason}\PYG{p}{,} \PYG{n+nv}{\PYGZdl{}index}\PYG{p}{)} \PYG{p}{\PYGZob{}}
        \PYG{c+c1}{// this is delivered each failed request}
    \PYG{p}{\PYGZcb{},}
\PYG{p}{]);}

\PYG{c+c1}{// Initiate the transfers and create a promise}
\PYG{n+nv}{\PYGZdl{}promise} \PYG{o}{=} \PYG{n+nv}{\PYGZdl{}pool}\PYG{o}{\PYGZhy{}\PYGZgt{}}\PYG{n+na}{promise}\PYG{p}{();}

\PYG{c+c1}{// Force the pool of requests to complete.}
\PYG{n+nv}{\PYGZdl{}promise}\PYG{o}{\PYGZhy{}\PYGZgt{}}\PYG{n+na}{wait}\PYG{p}{();}
\end{sphinxVerbatim}

Or using a closure that will return a promise once the pool calls the closure.

\begin{sphinxVerbatim}[commandchars=\\\{\}]
\PYG{n+nv}{\PYGZdl{}client} \PYG{o}{=} \PYG{k}{new} \PYG{n+nx}{Client}\PYG{p}{();}

\PYG{n+nv}{\PYGZdl{}requests} \PYG{o}{=} \PYG{k}{function} \PYG{p}{(}\PYG{n+nv}{\PYGZdl{}total}\PYG{p}{)} \PYG{k}{use} \PYG{p}{(}\PYG{n+nv}{\PYGZdl{}client}\PYG{p}{)} \PYG{p}{\PYGZob{}}
    \PYG{n+nv}{\PYGZdl{}uri} \PYG{o}{=} \PYG{l+s+s1}{\PYGZsq{}http://127.0.0.1:8126/guzzle\PYGZhy{}server/perf\PYGZsq{}}\PYG{p}{;}
    \PYG{k}{for} \PYG{p}{(}\PYG{n+nv}{\PYGZdl{}i} \PYG{o}{=} \PYG{l+m+mi}{0}\PYG{p}{;} \PYG{n+nv}{\PYGZdl{}i} \PYG{o}{\PYGZlt{}} \PYG{n+nv}{\PYGZdl{}total}\PYG{p}{;} \PYG{n+nv}{\PYGZdl{}i}\PYG{o}{++}\PYG{p}{)} \PYG{p}{\PYGZob{}}
        \PYG{k}{yield} \PYG{k}{function}\PYG{p}{()} \PYG{k}{use} \PYG{p}{(}\PYG{n+nv}{\PYGZdl{}client}\PYG{p}{,} \PYG{n+nv}{\PYGZdl{}uri}\PYG{p}{)} \PYG{p}{\PYGZob{}}
            \PYG{k}{return} \PYG{n+nv}{\PYGZdl{}client}\PYG{o}{\PYGZhy{}\PYGZgt{}}\PYG{n+na}{getAsync}\PYG{p}{(}\PYG{n+nv}{\PYGZdl{}uri}\PYG{p}{);}
        \PYG{p}{\PYGZcb{};}
    \PYG{p}{\PYGZcb{}}
\PYG{p}{\PYGZcb{};}

\PYG{n+nv}{\PYGZdl{}pool} \PYG{o}{=} \PYG{k}{new} \PYG{n+nx}{Pool}\PYG{p}{(}\PYG{n+nv}{\PYGZdl{}client}\PYG{p}{,} \PYG{n+nv}{\PYGZdl{}requests}\PYG{p}{(}\PYG{l+m+mi}{100}\PYG{p}{));}
\end{sphinxVerbatim}


\chapter{Using Responses}
\label{\detokenize{quickstart:using-responses}}
In the previous examples, we retrieved a \sphinxcode{\$response} variable or we were
delivered a response from a promise. The response object implements a PSR-7
response, \sphinxcode{Psr\textbackslash{}Http\textbackslash{}Message\textbackslash{}ResponseInterface}, and contains lots of
helpful information.

You can get the status code and reason phrase of the response:

\begin{sphinxVerbatim}[commandchars=\\\{\}]
\PYG{n+nv}{\PYGZdl{}code} \PYG{o}{=} \PYG{n+nv}{\PYGZdl{}response}\PYG{o}{\PYGZhy{}\PYGZgt{}}\PYG{n+na}{getStatusCode}\PYG{p}{();} \PYG{c+c1}{// 200}
\PYG{n+nv}{\PYGZdl{}reason} \PYG{o}{=} \PYG{n+nv}{\PYGZdl{}response}\PYG{o}{\PYGZhy{}\PYGZgt{}}\PYG{n+na}{getReasonPhrase}\PYG{p}{();} \PYG{c+c1}{// OK}
\end{sphinxVerbatim}

You can retrieve headers from the response:

\begin{sphinxVerbatim}[commandchars=\\\{\}]
\PYG{c+c1}{// Check if a header exists.}
\PYG{k}{if} \PYG{p}{(}\PYG{n+nv}{\PYGZdl{}response}\PYG{o}{\PYGZhy{}\PYGZgt{}}\PYG{n+na}{hasHeader}\PYG{p}{(}\PYG{l+s+s1}{\PYGZsq{}Content\PYGZhy{}Length\PYGZsq{}}\PYG{p}{))} \PYG{p}{\PYGZob{}}
    \PYG{k}{echo} \PYG{l+s+s2}{\PYGZdq{}}\PYG{l+s+s2}{It exists}\PYG{l+s+s2}{\PYGZdq{}}\PYG{p}{;}
\PYG{p}{\PYGZcb{}}

\PYG{c+c1}{// Get a header from the response.}
\PYG{k}{echo} \PYG{n+nv}{\PYGZdl{}response}\PYG{o}{\PYGZhy{}\PYGZgt{}}\PYG{n+na}{getHeader}\PYG{p}{(}\PYG{l+s+s1}{\PYGZsq{}Content\PYGZhy{}Length\PYGZsq{}}\PYG{p}{);}

\PYG{c+c1}{// Get all of the response headers.}
\PYG{k}{foreach} \PYG{p}{(}\PYG{n+nv}{\PYGZdl{}response}\PYG{o}{\PYGZhy{}\PYGZgt{}}\PYG{n+na}{getHeaders}\PYG{p}{()} \PYG{k}{as} \PYG{n+nv}{\PYGZdl{}name} \PYG{o}{=\PYGZgt{}} \PYG{n+nv}{\PYGZdl{}values}\PYG{p}{)} \PYG{p}{\PYGZob{}}
    \PYG{k}{echo} \PYG{n+nv}{\PYGZdl{}name} \PYG{o}{.} \PYG{l+s+s1}{\PYGZsq{}: \PYGZsq{}} \PYG{o}{.} \PYG{n+nb}{implode}\PYG{p}{(}\PYG{l+s+s1}{\PYGZsq{}, \PYGZsq{}}\PYG{p}{,} \PYG{n+nv}{\PYGZdl{}values}\PYG{p}{)} \PYG{o}{.} \PYG{l+s+s2}{\PYGZdq{}}\PYG{l+s+se}{\PYGZbs{}r}\PYG{l+s+se}{\PYGZbs{}n}\PYG{l+s+s2}{\PYGZdq{}}\PYG{p}{;}
\PYG{p}{\PYGZcb{}}
\end{sphinxVerbatim}

The body of a response can be retrieved using the \sphinxcode{getBody} method. The body
can be used as a string, cast to a string, or used as a stream like object.

\begin{sphinxVerbatim}[commandchars=\\\{\}]
\PYG{n+nv}{\PYGZdl{}body} \PYG{o}{=} \PYG{n+nv}{\PYGZdl{}response}\PYG{o}{\PYGZhy{}\PYGZgt{}}\PYG{n+na}{getBody}\PYG{p}{();}
\PYG{c+c1}{// Implicitly cast the body to a string and echo it}
\PYG{k}{echo} \PYG{n+nv}{\PYGZdl{}body}\PYG{p}{;}
\PYG{c+c1}{// Explicitly cast the body to a string}
\PYG{n+nv}{\PYGZdl{}stringBody} \PYG{o}{=} \PYG{p}{(}\PYG{n+nx}{string}\PYG{p}{)} \PYG{n+nv}{\PYGZdl{}body}\PYG{p}{;}
\PYG{c+c1}{// Read 10 bytes from the body}
\PYG{n+nv}{\PYGZdl{}tenBytes} \PYG{o}{=} \PYG{n+nv}{\PYGZdl{}body}\PYG{o}{\PYGZhy{}\PYGZgt{}}\PYG{n+na}{read}\PYG{p}{(}\PYG{l+m+mi}{10}\PYG{p}{);}
\PYG{c+c1}{// Read the remaining contents of the body as a string}
\PYG{n+nv}{\PYGZdl{}remainingBytes} \PYG{o}{=} \PYG{n+nv}{\PYGZdl{}body}\PYG{o}{\PYGZhy{}\PYGZgt{}}\PYG{n+na}{getContents}\PYG{p}{();}
\end{sphinxVerbatim}


\chapter{Query String Parameters}
\label{\detokenize{quickstart:query-string-parameters}}
You can provide query string parameters with a request in several ways.

You can set query string parameters in the request's URI:

\begin{sphinxVerbatim}[commandchars=\\\{\}]
\PYG{n+nv}{\PYGZdl{}response} \PYG{o}{=} \PYG{n+nv}{\PYGZdl{}client}\PYG{o}{\PYGZhy{}\PYGZgt{}}\PYG{n+na}{request}\PYG{p}{(}\PYG{l+s+s1}{\PYGZsq{}GET\PYGZsq{}}\PYG{p}{,} \PYG{l+s+s1}{\PYGZsq{}http://httpbin.org?foo=bar\PYGZsq{}}\PYG{p}{);}
\end{sphinxVerbatim}

You can specify the query string parameters using the \sphinxcode{query} request
option as an array.

\begin{sphinxVerbatim}[commandchars=\\\{\}]
\PYG{n+nv}{\PYGZdl{}client}\PYG{o}{\PYGZhy{}\PYGZgt{}}\PYG{n+na}{request}\PYG{p}{(}\PYG{l+s+s1}{\PYGZsq{}GET\PYGZsq{}}\PYG{p}{,} \PYG{l+s+s1}{\PYGZsq{}http://httpbin.org\PYGZsq{}}\PYG{p}{,} \PYG{p}{[}
    \PYG{l+s+s1}{\PYGZsq{}query\PYGZsq{}} \PYG{o}{=\PYGZgt{}} \PYG{p}{[}\PYG{l+s+s1}{\PYGZsq{}foo\PYGZsq{}} \PYG{o}{=\PYGZgt{}} \PYG{l+s+s1}{\PYGZsq{}bar\PYGZsq{}}\PYG{p}{]}
\PYG{p}{]);}
\end{sphinxVerbatim}

Providing the option as an array will use PHP's \sphinxcode{http\_build\_query} function
to format the query string.

And finally, you can provide the \sphinxcode{query} request option as a string.

\begin{sphinxVerbatim}[commandchars=\\\{\}]
\PYG{n+nv}{\PYGZdl{}client}\PYG{o}{\PYGZhy{}\PYGZgt{}}\PYG{n+na}{request}\PYG{p}{(}\PYG{l+s+s1}{\PYGZsq{}GET\PYGZsq{}}\PYG{p}{,} \PYG{l+s+s1}{\PYGZsq{}http://httpbin.org\PYGZsq{}}\PYG{p}{,} \PYG{p}{[}\PYG{l+s+s1}{\PYGZsq{}query\PYGZsq{}} \PYG{o}{=\PYGZgt{}} \PYG{l+s+s1}{\PYGZsq{}foo=bar\PYGZsq{}}\PYG{p}{]);}
\end{sphinxVerbatim}


\chapter{Uploading Data}
\label{\detokenize{quickstart:uploading-data}}
Guzzle provides several methods for uploading data.

You can send requests that contain a stream of data by passing a string,
resource returned from \sphinxcode{fopen}, or an instance of a
\sphinxcode{Psr\textbackslash{}Http\textbackslash{}Message\textbackslash{}StreamInterface} to the \sphinxcode{body} request option.

\begin{sphinxVerbatim}[commandchars=\\\{\}]
\PYG{c+c1}{// Provide the body as a string.}
\PYG{n+nv}{\PYGZdl{}r} \PYG{o}{=} \PYG{n+nv}{\PYGZdl{}client}\PYG{o}{\PYGZhy{}\PYGZgt{}}\PYG{n+na}{request}\PYG{p}{(}\PYG{l+s+s1}{\PYGZsq{}POST\PYGZsq{}}\PYG{p}{,} \PYG{l+s+s1}{\PYGZsq{}http://httpbin.org/post\PYGZsq{}}\PYG{p}{,} \PYG{p}{[}
    \PYG{l+s+s1}{\PYGZsq{}body\PYGZsq{}} \PYG{o}{=\PYGZgt{}} \PYG{l+s+s1}{\PYGZsq{}raw data\PYGZsq{}}
\PYG{p}{]);}

\PYG{c+c1}{// Provide an fopen resource.}
\PYG{n+nv}{\PYGZdl{}body} \PYG{o}{=} \PYG{n+nb}{fopen}\PYG{p}{(}\PYG{l+s+s1}{\PYGZsq{}/path/to/file\PYGZsq{}}\PYG{p}{,} \PYG{l+s+s1}{\PYGZsq{}r\PYGZsq{}}\PYG{p}{);}
\PYG{n+nv}{\PYGZdl{}r} \PYG{o}{=} \PYG{n+nv}{\PYGZdl{}client}\PYG{o}{\PYGZhy{}\PYGZgt{}}\PYG{n+na}{request}\PYG{p}{(}\PYG{l+s+s1}{\PYGZsq{}POST\PYGZsq{}}\PYG{p}{,} \PYG{l+s+s1}{\PYGZsq{}http://httpbin.org/post\PYGZsq{}}\PYG{p}{,} \PYG{p}{[}\PYG{l+s+s1}{\PYGZsq{}body\PYGZsq{}} \PYG{o}{=\PYGZgt{}} \PYG{n+nv}{\PYGZdl{}body}\PYG{p}{]);}

\PYG{c+c1}{// Use the stream\PYGZus{}for() function to create a PSR\PYGZhy{}7 stream.}
\PYG{n+nv}{\PYGZdl{}body} \PYG{o}{=} \PYG{n+nx}{\PYGZbs{}GuzzleHttp\PYGZbs{}Psr7\PYGZbs{}stream\PYGZus{}for}\PYG{p}{(}\PYG{l+s+s1}{\PYGZsq{}hello!\PYGZsq{}}\PYG{p}{);}
\PYG{n+nv}{\PYGZdl{}r} \PYG{o}{=} \PYG{n+nv}{\PYGZdl{}client}\PYG{o}{\PYGZhy{}\PYGZgt{}}\PYG{n+na}{request}\PYG{p}{(}\PYG{l+s+s1}{\PYGZsq{}POST\PYGZsq{}}\PYG{p}{,} \PYG{l+s+s1}{\PYGZsq{}http://httpbin.org/post\PYGZsq{}}\PYG{p}{,} \PYG{p}{[}\PYG{l+s+s1}{\PYGZsq{}body\PYGZsq{}} \PYG{o}{=\PYGZgt{}} \PYG{n+nv}{\PYGZdl{}body}\PYG{p}{]);}
\end{sphinxVerbatim}

An easy way to upload JSON data and set the appropriate header is using the
\sphinxcode{json} request option:

\begin{sphinxVerbatim}[commandchars=\\\{\}]
\PYG{n+nv}{\PYGZdl{}r} \PYG{o}{=} \PYG{n+nv}{\PYGZdl{}client}\PYG{o}{\PYGZhy{}\PYGZgt{}}\PYG{n+na}{request}\PYG{p}{(}\PYG{l+s+s1}{\PYGZsq{}PUT\PYGZsq{}}\PYG{p}{,} \PYG{l+s+s1}{\PYGZsq{}http://httpbin.org/put\PYGZsq{}}\PYG{p}{,} \PYG{p}{[}
    \PYG{l+s+s1}{\PYGZsq{}json\PYGZsq{}} \PYG{o}{=\PYGZgt{}} \PYG{p}{[}\PYG{l+s+s1}{\PYGZsq{}foo\PYGZsq{}} \PYG{o}{=\PYGZgt{}} \PYG{l+s+s1}{\PYGZsq{}bar\PYGZsq{}}\PYG{p}{]}
\PYG{p}{]);}
\end{sphinxVerbatim}


\section{POST/Form Requests}
\label{\detokenize{quickstart:post-form-requests}}
In addition to specifying the raw data of a request using the \sphinxcode{body} request
option, Guzzle provides helpful abstractions over sending POST data.


\subsection{Sending form fields}
\label{\detokenize{quickstart:sending-form-fields}}
Sending \sphinxcode{application/x-www-form-urlencoded} POST requests requires that you
specify the POST fields as an array in the \sphinxcode{form\_params} request options.

\begin{sphinxVerbatim}[commandchars=\\\{\}]
\PYG{n+nv}{\PYGZdl{}response} \PYG{o}{=} \PYG{n+nv}{\PYGZdl{}client}\PYG{o}{\PYGZhy{}\PYGZgt{}}\PYG{n+na}{request}\PYG{p}{(}\PYG{l+s+s1}{\PYGZsq{}POST\PYGZsq{}}\PYG{p}{,} \PYG{l+s+s1}{\PYGZsq{}http://httpbin.org/post\PYGZsq{}}\PYG{p}{,} \PYG{p}{[}
    \PYG{l+s+s1}{\PYGZsq{}form\PYGZus{}params\PYGZsq{}} \PYG{o}{=\PYGZgt{}} \PYG{p}{[}
        \PYG{l+s+s1}{\PYGZsq{}field\PYGZus{}name\PYGZsq{}} \PYG{o}{=\PYGZgt{}} \PYG{l+s+s1}{\PYGZsq{}abc\PYGZsq{}}\PYG{p}{,}
        \PYG{l+s+s1}{\PYGZsq{}other\PYGZus{}field\PYGZsq{}} \PYG{o}{=\PYGZgt{}} \PYG{l+s+s1}{\PYGZsq{}123\PYGZsq{}}\PYG{p}{,}
        \PYG{l+s+s1}{\PYGZsq{}nested\PYGZus{}field\PYGZsq{}} \PYG{o}{=\PYGZgt{}} \PYG{p}{[}
            \PYG{l+s+s1}{\PYGZsq{}nested\PYGZsq{}} \PYG{o}{=\PYGZgt{}} \PYG{l+s+s1}{\PYGZsq{}hello\PYGZsq{}}
        \PYG{p}{]}
    \PYG{p}{]}
\PYG{p}{]);}
\end{sphinxVerbatim}


\subsection{Sending form files}
\label{\detokenize{quickstart:sending-form-files}}
You can send files along with a form (\sphinxcode{multipart/form-data} POST requests),
using the \sphinxcode{multipart} request option. \sphinxcode{multipart} accepts an array of
associative arrays, where each associative array contains the following keys:
\begin{itemize}
\item {} 
name: (required, string) key mapping to the form field name.

\item {} 
contents: (required, mixed) Provide a string to send the contents of the
file as a string, provide an fopen resource to stream the contents from a
PHP stream, or provide a \sphinxcode{Psr\textbackslash{}Http\textbackslash{}Message\textbackslash{}StreamInterface} to stream
the contents from a PSR-7 stream.

\end{itemize}

\begin{sphinxVerbatim}[commandchars=\\\{\}]
\PYG{n+nv}{\PYGZdl{}response} \PYG{o}{=} \PYG{n+nv}{\PYGZdl{}client}\PYG{o}{\PYGZhy{}\PYGZgt{}}\PYG{n+na}{request}\PYG{p}{(}\PYG{l+s+s1}{\PYGZsq{}POST\PYGZsq{}}\PYG{p}{,} \PYG{l+s+s1}{\PYGZsq{}http://httpbin.org/post\PYGZsq{}}\PYG{p}{,} \PYG{p}{[}
    \PYG{l+s+s1}{\PYGZsq{}multipart\PYGZsq{}} \PYG{o}{=\PYGZgt{}} \PYG{p}{[}
        \PYG{p}{[}
            \PYG{l+s+s1}{\PYGZsq{}name\PYGZsq{}}     \PYG{o}{=\PYGZgt{}} \PYG{l+s+s1}{\PYGZsq{}field\PYGZus{}name\PYGZsq{}}\PYG{p}{,}
            \PYG{l+s+s1}{\PYGZsq{}contents\PYGZsq{}} \PYG{o}{=\PYGZgt{}} \PYG{l+s+s1}{\PYGZsq{}abc\PYGZsq{}}
        \PYG{p}{],}
        \PYG{p}{[}
            \PYG{l+s+s1}{\PYGZsq{}name\PYGZsq{}}     \PYG{o}{=\PYGZgt{}} \PYG{l+s+s1}{\PYGZsq{}file\PYGZus{}name\PYGZsq{}}\PYG{p}{,}
            \PYG{l+s+s1}{\PYGZsq{}contents\PYGZsq{}} \PYG{o}{=\PYGZgt{}} \PYG{n+nb}{fopen}\PYG{p}{(}\PYG{l+s+s1}{\PYGZsq{}/path/to/file\PYGZsq{}}\PYG{p}{,} \PYG{l+s+s1}{\PYGZsq{}r\PYGZsq{}}\PYG{p}{)}
        \PYG{p}{],}
        \PYG{p}{[}
            \PYG{l+s+s1}{\PYGZsq{}name\PYGZsq{}}     \PYG{o}{=\PYGZgt{}} \PYG{l+s+s1}{\PYGZsq{}other\PYGZus{}file\PYGZsq{}}\PYG{p}{,}
            \PYG{l+s+s1}{\PYGZsq{}contents\PYGZsq{}} \PYG{o}{=\PYGZgt{}} \PYG{l+s+s1}{\PYGZsq{}hello\PYGZsq{}}\PYG{p}{,}
            \PYG{l+s+s1}{\PYGZsq{}filename\PYGZsq{}} \PYG{o}{=\PYGZgt{}} \PYG{l+s+s1}{\PYGZsq{}filename.txt\PYGZsq{}}\PYG{p}{,}
            \PYG{l+s+s1}{\PYGZsq{}headers\PYGZsq{}}  \PYG{o}{=\PYGZgt{}} \PYG{p}{[}
                \PYG{l+s+s1}{\PYGZsq{}X\PYGZhy{}Foo\PYGZsq{}} \PYG{o}{=\PYGZgt{}} \PYG{l+s+s1}{\PYGZsq{}this is an extra header to include\PYGZsq{}}
            \PYG{p}{]}
        \PYG{p}{]}
    \PYG{p}{]}
\PYG{p}{]);}
\end{sphinxVerbatim}


\chapter{Cookies}
\label{\detokenize{quickstart:cookies}}
Guzzle can maintain a cookie session for you if instructed using the
\sphinxcode{cookies} request option. When sending a request, the \sphinxcode{cookies} option
must be set to an instance of \sphinxcode{GuzzleHttp\textbackslash{}Cookie\textbackslash{}CookieJarInterface}.

\begin{sphinxVerbatim}[commandchars=\\\{\}]
\PYG{c+c1}{// Use a specific cookie jar}
\PYG{n+nv}{\PYGZdl{}jar} \PYG{o}{=} \PYG{k}{new} \PYG{n+nx}{\PYGZbs{}GuzzleHttp\PYGZbs{}Cookie\PYGZbs{}CookieJar}\PYG{p}{;}
\PYG{n+nv}{\PYGZdl{}r} \PYG{o}{=} \PYG{n+nv}{\PYGZdl{}client}\PYG{o}{\PYGZhy{}\PYGZgt{}}\PYG{n+na}{request}\PYG{p}{(}\PYG{l+s+s1}{\PYGZsq{}GET\PYGZsq{}}\PYG{p}{,} \PYG{l+s+s1}{\PYGZsq{}http://httpbin.org/cookies\PYGZsq{}}\PYG{p}{,} \PYG{p}{[}
    \PYG{l+s+s1}{\PYGZsq{}cookies\PYGZsq{}} \PYG{o}{=\PYGZgt{}} \PYG{n+nv}{\PYGZdl{}jar}
\PYG{p}{]);}
\end{sphinxVerbatim}

You can set \sphinxcode{cookies} to \sphinxcode{true} in a client constructor if you would like
to use a shared cookie jar for all requests.

\begin{sphinxVerbatim}[commandchars=\\\{\}]
\PYG{c+c1}{// Use a shared client cookie jar}
\PYG{n+nv}{\PYGZdl{}client} \PYG{o}{=} \PYG{k}{new} \PYG{n+nx}{\PYGZbs{}GuzzleHttp\PYGZbs{}Client}\PYG{p}{([}\PYG{l+s+s1}{\PYGZsq{}cookies\PYGZsq{}} \PYG{o}{=\PYGZgt{}} \PYG{k}{true}\PYG{p}{]);}
\PYG{n+nv}{\PYGZdl{}r} \PYG{o}{=} \PYG{n+nv}{\PYGZdl{}client}\PYG{o}{\PYGZhy{}\PYGZgt{}}\PYG{n+na}{request}\PYG{p}{(}\PYG{l+s+s1}{\PYGZsq{}GET\PYGZsq{}}\PYG{p}{,} \PYG{l+s+s1}{\PYGZsq{}http://httpbin.org/cookies\PYGZsq{}}\PYG{p}{);}
\end{sphinxVerbatim}


\chapter{Redirects}
\label{\detokenize{quickstart:redirects}}
Guzzle will automatically follow redirects unless you tell it not to. You can
customize the redirect behavior using the \sphinxcode{allow\_redirects} request option.
\begin{itemize}
\item {} 
Set to \sphinxcode{true} to enable normal redirects with a maximum number of 5
redirects. This is the default setting.

\item {} 
Set to \sphinxcode{false} to disable redirects.

\item {} 
Pass an associative array containing the `max' key to specify the maximum
number of redirects and optionally provide a `strict' key value to specify
whether or not to use strict RFC compliant redirects (meaning redirect POST
requests with POST requests vs. doing what most browsers do which is
redirect POST requests with GET requests).

\end{itemize}

\begin{sphinxVerbatim}[commandchars=\\\{\}]
\PYG{n+nv}{\PYGZdl{}response} \PYG{o}{=} \PYG{n+nv}{\PYGZdl{}client}\PYG{o}{\PYGZhy{}\PYGZgt{}}\PYG{n+na}{request}\PYG{p}{(}\PYG{l+s+s1}{\PYGZsq{}GET\PYGZsq{}}\PYG{p}{,} \PYG{l+s+s1}{\PYGZsq{}http://github.com\PYGZsq{}}\PYG{p}{);}
\PYG{k}{echo} \PYG{n+nv}{\PYGZdl{}response}\PYG{o}{\PYGZhy{}\PYGZgt{}}\PYG{n+na}{getStatusCode}\PYG{p}{();}
\PYG{c+c1}{// 200}
\end{sphinxVerbatim}

The following example shows that redirects can be disabled.

\begin{sphinxVerbatim}[commandchars=\\\{\}]
\PYG{n+nv}{\PYGZdl{}response} \PYG{o}{=} \PYG{n+nv}{\PYGZdl{}client}\PYG{o}{\PYGZhy{}\PYGZgt{}}\PYG{n+na}{request}\PYG{p}{(}\PYG{l+s+s1}{\PYGZsq{}GET\PYGZsq{}}\PYG{p}{,} \PYG{l+s+s1}{\PYGZsq{}http://github.com\PYGZsq{}}\PYG{p}{,} \PYG{p}{[}
    \PYG{l+s+s1}{\PYGZsq{}allow\PYGZus{}redirects\PYGZsq{}} \PYG{o}{=\PYGZgt{}} \PYG{k}{false}
\PYG{p}{]);}
\PYG{k}{echo} \PYG{n+nv}{\PYGZdl{}response}\PYG{o}{\PYGZhy{}\PYGZgt{}}\PYG{n+na}{getStatusCode}\PYG{p}{();}
\PYG{c+c1}{// 301}
\end{sphinxVerbatim}


\chapter{Exceptions}
\label{\detokenize{quickstart:exceptions}}
Guzzle throws exceptions for errors that occur during a transfer.
\begin{itemize}
\item {} 
In the event of a networking error (connection timeout, DNS errors, etc.),
a \sphinxcode{GuzzleHttp\textbackslash{}Exception\textbackslash{}RequestException} is thrown. This exception
extends from \sphinxcode{GuzzleHttp\textbackslash{}Exception\textbackslash{}TransferException}. Catching this
exception will catch any exception that can be thrown while transferring
requests.

\begin{sphinxVerbatim}[commandchars=\\\{\}]
\PYG{k}{use} \PYG{n+nx}{GuzzleHttp\PYGZbs{}Psr7}\PYG{p}{;}
\PYG{k}{use} \PYG{n+nx}{GuzzleHttp\PYGZbs{}Exception\PYGZbs{}RequestException}\PYG{p}{;}

\PYG{k}{try} \PYG{p}{\PYGZob{}}
    \PYG{n+nv}{\PYGZdl{}client}\PYG{o}{\PYGZhy{}\PYGZgt{}}\PYG{n+na}{request}\PYG{p}{(}\PYG{l+s+s1}{\PYGZsq{}GET\PYGZsq{}}\PYG{p}{,} \PYG{l+s+s1}{\PYGZsq{}https://github.com/\PYGZus{}abc\PYGZus{}123\PYGZus{}404\PYGZsq{}}\PYG{p}{);}
\PYG{p}{\PYGZcb{}} \PYG{k}{catch} \PYG{p}{(}\PYG{n+nx}{RequestException} \PYG{n+nv}{\PYGZdl{}e}\PYG{p}{)} \PYG{p}{\PYGZob{}}
    \PYG{k}{echo} \PYG{n+nx}{Psr7\PYGZbs{}str}\PYG{p}{(}\PYG{n+nv}{\PYGZdl{}e}\PYG{o}{\PYGZhy{}\PYGZgt{}}\PYG{n+na}{getRequest}\PYG{p}{());}
    \PYG{k}{if} \PYG{p}{(}\PYG{n+nv}{\PYGZdl{}e}\PYG{o}{\PYGZhy{}\PYGZgt{}}\PYG{n+na}{hasResponse}\PYG{p}{())} \PYG{p}{\PYGZob{}}
        \PYG{k}{echo} \PYG{n+nx}{Psr7\PYGZbs{}str}\PYG{p}{(}\PYG{n+nv}{\PYGZdl{}e}\PYG{o}{\PYGZhy{}\PYGZgt{}}\PYG{n+na}{getResponse}\PYG{p}{());}
    \PYG{p}{\PYGZcb{}}
\PYG{p}{\PYGZcb{}}
\end{sphinxVerbatim}

\item {} 
A \sphinxcode{GuzzleHttp\textbackslash{}Exception\textbackslash{}ConnectException} exception is thrown in the
event of a networking error. This exception extends from
\sphinxcode{GuzzleHttp\textbackslash{}Exception\textbackslash{}RequestException}.

\item {} 
A \sphinxcode{GuzzleHttp\textbackslash{}Exception\textbackslash{}ClientException} is thrown for 400
level errors if the \sphinxcode{http\_errors} request option is set to true. This
exception extends from \sphinxcode{GuzzleHttp\textbackslash{}Exception\textbackslash{}BadResponseException} and
\sphinxcode{GuzzleHttp\textbackslash{}Exception\textbackslash{}BadResponseException} extends from
\sphinxcode{GuzzleHttp\textbackslash{}Exception\textbackslash{}RequestException}.

\begin{sphinxVerbatim}[commandchars=\\\{\}]
\PYG{k}{use} \PYG{n+nx}{GuzzleHttp\PYGZbs{}Exception\PYGZbs{}ClientException}\PYG{p}{;}

\PYG{k}{try} \PYG{p}{\PYGZob{}}
    \PYG{n+nv}{\PYGZdl{}client}\PYG{o}{\PYGZhy{}\PYGZgt{}}\PYG{n+na}{request}\PYG{p}{(}\PYG{l+s+s1}{\PYGZsq{}GET\PYGZsq{}}\PYG{p}{,} \PYG{l+s+s1}{\PYGZsq{}https://github.com/\PYGZus{}abc\PYGZus{}123\PYGZus{}404\PYGZsq{}}\PYG{p}{);}
\PYG{p}{\PYGZcb{}} \PYG{k}{catch} \PYG{p}{(}\PYG{n+nx}{ClientException} \PYG{n+nv}{\PYGZdl{}e}\PYG{p}{)} \PYG{p}{\PYGZob{}}
    \PYG{k}{echo} \PYG{n+nx}{Psr7\PYGZbs{}str}\PYG{p}{(}\PYG{n+nv}{\PYGZdl{}e}\PYG{o}{\PYGZhy{}\PYGZgt{}}\PYG{n+na}{getRequest}\PYG{p}{());}
    \PYG{k}{echo} \PYG{n+nx}{Psr7\PYGZbs{}str}\PYG{p}{(}\PYG{n+nv}{\PYGZdl{}e}\PYG{o}{\PYGZhy{}\PYGZgt{}}\PYG{n+na}{getResponse}\PYG{p}{());}
\PYG{p}{\PYGZcb{}}
\end{sphinxVerbatim}

\item {} 
A \sphinxcode{GuzzleHttp\textbackslash{}Exception\textbackslash{}ServerException} is thrown for 500 level
errors if the \sphinxcode{http\_errors} request option is set to true. This
exception extends from \sphinxcode{GuzzleHttp\textbackslash{}Exception\textbackslash{}BadResponseException}.

\item {} 
A \sphinxcode{GuzzleHttp\textbackslash{}Exception\textbackslash{}TooManyRedirectsException} is thrown when too
many redirects are followed. This exception extends from \sphinxcode{GuzzleHttp\textbackslash{}Exception\textbackslash{}RequestException}.

\end{itemize}

All of the above exceptions extend from
\sphinxcode{GuzzleHttp\textbackslash{}Exception\textbackslash{}TransferException}.


\chapter{Environment Variables}
\label{\detokenize{quickstart:environment-variables}}
Guzzle exposes a few environment variables that can be used to customize the
behavior of the library.
\begin{description}
\item[{\sphinxcode{GUZZLE\_CURL\_SELECT\_TIMEOUT}}] \leavevmode
Controls the duration in seconds that a curl\_multi\_* handler will use when
selecting on curl handles using \sphinxcode{curl\_multi\_select()}. Some systems
have issues with PHP's implementation of \sphinxcode{curl\_multi\_select()} where
calling this function always results in waiting for the maximum duration of
the timeout.

\item[{\sphinxcode{HTTP\_PROXY}}] \leavevmode
Defines the proxy to use when sending requests using the ``http'' protocol.

Note: because the HTTP\_PROXY variable may contain arbitrary user input on some (CGI) environments, the variable is only used on the CLI SAPI. See \url{https://httpoxy.org} for more information.

\item[{\sphinxcode{HTTPS\_PROXY}}] \leavevmode
Defines the proxy to use when sending requests using the ``https'' protocol.

\end{description}


\section{Relevant ini Settings}
\label{\detokenize{quickstart:relevant-ini-settings}}
Guzzle can utilize PHP ini settings when configuring clients.
\begin{description}
\item[{\sphinxcode{openssl.cafile}}] \leavevmode
Specifies the path on disk to a CA file in PEM format to use when sending
requests over ``https''. See: \url{https://wiki.php.net/rfc/tls-peer-verification\#phpini\_defaults}

\end{description}


\part{Indices and tables}
\label{\detokenize{index:indices-and-tables}}\begin{itemize}
\item {} 
\DUrole{xref,std,std-ref}{genindex}

\item {} 
\DUrole{xref,std,std-ref}{modindex}

\item {} 
\DUrole{xref,std,std-ref}{search}

\end{itemize}



\renewcommand{\indexname}{Índice}
\printindex
\end{document}